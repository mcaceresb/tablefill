%%%%%%%%%%%%%%%%%%%%%%%%%%%%%%%%%%%%%%%%%%%%%%%%%%%%%%%%%%%%%%%%%%%%%%%%
%%%%%%%%%%%%%%%%%%%%%%%%%%%%%%%%%%%%%%%%%%%%%%%%%%%%%%%%%%%%%%%%%%%%%%%%
%%%%%%%%%%%%%%%%%%%%%%%%%%%%%%%%%%%%%%%%%%%%%%%%%%%%%%%%%%%%%%%%%%%%%%%%
% This file was produced by 'tablefill.py'
% 	Template file: /home/mauricio/Documents/projects/dev/code/archive/2015/tablefill/docs/usage/01basic/template.tex
% 	Input file(s): ['/home/mauricio/Documents/projects/dev/code/archive/2015/tablefill/docs/usage/01basic/input.txt']
% To make changes, edit the input and template files.
% % 

% DO NOT EDIT THIS FILE DIRECTLY.
%%%%%%%%%%%%%%%%%%%%%%%%%%%%%%%%%%%%%%%%%%%%%%%%%%%%%%%%%%%%%%%%%%%%%%%%
%%%%%%%%%%%%%%%%%%%%%%%%%%%%%%%%%%%%%%%%%%%%%%%%%%%%%%%%%%%%%%%%%%%%%%%%
%%%%%%%%%%%%%%%%%%%%%%%%%%%%%%%%%%%%%%%%%%%%%%%%%%%%%%%%%%%%%%%%%%%%%%%%
\documentclass{article}
\usepackage{booktabs}
\begin{document}

Tablefill will look for the label \verb'tab:example' inside
\verb'input.txt' and fill the table below:

\begin{table}
  \caption{Table caption (e.g. summary stats)}
  \label{tab:example} % name must match label in input1.txt
  \begin{tabular}{p{4.25cm}ccc}
    \toprule
    Outcomes
    & N
    & Mean
    & (Std.)
    \\\midrule
    Outcomes 1 & 1,237 & 1.0 & (2.00) \\
    Outcomes 2 & 2,234 & 3.0 & (2.40) \\
    Outcomes 3 & 3 & 2.0 & (2.46) \\
    Outcomes 4 & 2,234 & 3.0 & (2.40) \\
    \bottomrule
    \multicolumn{4}{p{5cm}}{\footnotesize Footnotes!}
  \end{tabular}
\end{table}

% tablefill:start tab:paragraph
Placeholders do not need to be inside a table. You can also have
placeholders in the text:
\begin{itemize}
    \item $N = 5,708$
    \item This is the 'tablefill example' sample.
\end{itemize}

Note \verb'% tablefill:start tab:paragraph' tells tablefill to start
looking for placeholders using the matrix labeled \verb'tab:paragraph'
in the input file. \verb'% tablefill:end' tells tablefill to stop.

Tablefill provides several placeholders types (more on this below),
but advanced users can use any format allowed by python via \verb'{}':
'python formatting'. Here is another table, using python-style formatting:
% tablefill:end

\begin{table}
  \caption{Table caption (e.g. regression results)}
  \label{tab:anotherExample} % must match label in input.txt
  \begin{tabular}{p{4.25cm}cc}
    \toprule
    Outcomes
    & Coef
    & (SE)
    \\\midrule
    Variable 1 & -1.2 & (-1.18) \\
    Variable 2 & 2.8 & (-0.53)*** \\
    Variable 3 & 1.1 & (0.57)** \\
             N & 5,708
    \\\midrule
    \bottomrule
    \multicolumn{3}{p{5cm}}{\footnotesize Footnotes!}
  \end{tabular}
\end{table}
\end{document}
